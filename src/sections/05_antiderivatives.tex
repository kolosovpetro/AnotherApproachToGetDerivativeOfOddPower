So far we have successfully reviewed and discussed an identity in terms of partial derivatives evaluated in
particular points $(v=u, z=u)$ that equals to the ordinary derivative of odd-power function $t^{2u+1}$,
see~\eqref{thm:main-theorem}.
However, what is about the inverse process of differentiation, i.e.\ antiderivatives?
Would theorem~\eqref{thm:main-theorem} have an analog in terms of antiderivatives?
Well, consider an example.
Heaving $m=2$ antiderivatives of $f_y(x,z)$ are
\begin{align*}
    \int f_y(x,z) \, dx &= \frac{1}{6} x z (-45 x z (1 + z)^2 + 10 x^2 (1 + 3 z + 2 z^2) +
    6 z^2 (10 + 15 z + 6 z^2))  + C_1 \\
    \int f_y(x,z) \, dz &= \frac{1}{2} z^2 (5 x^2 (1 + z)^2 + z^2 (5 + 6 z + 2 z^2) -
    x z (10 + 15 z + 6 z^2)) + C_2
\end{align*}
Let be
\begin{align*}
    F = \int f_y(x,z) \, dx + C_1 \\
    G = \int f_y(x,z) \, dz + C_2
\end{align*}
Then, evaluating $F$ and $G$ in points $(u,u)$ yields
\begin{align*}
    F(u, u) + G(u, u) = \frac{1}{6} 25 u^4 + \frac{1}{2} 11 u^5 + \frac{1}{3} 7 u^6 + C_1 + C_2
\end{align*}
Possible analog of theorem~\eqref{thm:main-theorem} assumes that
\begin{align*}
    F(u, u) + G(u, u) \equiv u^5
\end{align*}
Considering $C_1=C_2=C$ we get
\begin{align*}
    F(u, u) + G(u, u) = \frac{1}{6} 25 u^4 + \frac{1}{2} 11 u^5 + \frac{1}{3} 7 u^6 + 2C
\end{align*}
Thus, integration constant $C$ is evaluated as
\begin{align*}
    C = \frac{1}{12} (-25 u^4 - 27 u^5 - 14 u^6)
\end{align*}
Thus, antiderivative case of theorem~\eqref{thm:main-theorem} is a whole new direction for further research.
