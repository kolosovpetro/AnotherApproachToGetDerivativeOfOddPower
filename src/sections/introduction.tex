This manuscript provides another approach to get derivative of odd-power, that is approach based
on partial derivatives identity in terms of partial derivatives, extending the main idea explained in~\cite{kolosov_2022}
that gives polynomial identity in a form as follows
\begin{equation}
    n^{2m+1} = \sum_{k=1}^{n} \sum_{r=0}^{m} \coeffA{m}{r} k^r (n-k)^r
    \label{eq:odd-power-identity}
\end{equation}
where $m$ is fixed constant $m\in\mathbb{N}$, $n \in \mathbb{N}$ and $\coeffA{m}{r}$ are real coefficients defined
recursively, see~\cite{kolosov2016link}.
Define the function $f_{y}$ such that based on the identity~\eqref{eq:odd-power-identity} with the only difference that
values of $n, m$ in the right part of~\eqref{eq:odd-power-identity} appear to be parameters of the function $f_{y}$.
In contrast to the equation~\eqref{eq:odd-power-identity}, upper bound $n$ of the sum $\sum_{k=1}^{n}$ turned into fixed
function's parameter $y$ as well, so that $f_{y}$ defined as follows
\begin{definition} (Polynomial function $f_{y}$.)
    \begin{equation}
        f_{y} (x, z) = \sum_{k=1}^{z} \sum_{r=0}^{y} \coeffA{y}{r} k^r (x-k)^r
        \label{eq:definition-f}
    \end{equation}
\end{definition}
where $x, z\in \mathbb{R}$ and $y$ is constant $y \in \mathbb{N}$.
At first glance, equation~\eqref{eq:definition-f} might look complex, so in order to clarify
the function $f_y$ and polynomials it produces let's show few examples.
Substituting the values of $y=1,2,3$ to the function $f$ we get the following polynomials
\begin{align*}
    f_{1} (x, z) &= 3 x z - 3 z^2 + 3 x z^2 - 2 z^3 \\
    f_{2} (x, z) &= 5 x^2 z - 15 x z^2 + 15 x^2 z^2 + 10 z^3 - 30 x z^3 + 10 x^2 z^3 +
    15 z^4 - 15 x z^4 + 6 z^5 \\
    f_{3} (x, z) &= -7 x z + 14 x^2 z + 7 z^2 - 42 x z^2 + 35 x^3 z^2 + 28 z^3 - 140 x^2 z^3 + 70 x^3 z^3 + 175 x z^4 \\
    &- 210 x^2 z^4 + 35 x^3 z^4 - 70 z^5 + 210 x z^5 - 84 x^2 z^5 - 70 z^6 + 70 x z^6 - 20 z^7
\end{align*}
According to the main topic of the current manuscript, it provides Another approach to get derivative of odd-power.
Therefore, we define odd-power function we work in context of.
Odd-power function $g_y$ is a function defined as follows
\begin{definition}(Odd-power function $g_y$.)
    \begin{equation}
        g_{y}(x) = x^{2y + 1}
        \label{eq:definition-g}
    \end{equation}
\end{definition}
where $x\in \mathbb{R}$ and $y$ is constant $y\in \mathbb{N}$.
One more important thing is to conclude on partial derivative notation,
more precisely the following notation for the partial derivative is used across the manuscript and remains unchanged
\begin{notation} (Partial derivative.)
    Let be a function $f(x_1, x_2, \dots, x_n)$ defined over the real space $\mathbb{R}^n$.
    We denote partial derivative of the function $f$ with respect to $x_i$ as follows
    \begin{equation*}
        f^{'}_{x_i} = \lim_{\Delta x_i \to 0} \frac{f(x_1, x_2, \dots, x_i + \Delta x_i, \dots, x_n) - f(x_1, x_2, \dots, x_n)}{\Delta x_i}
    \end{equation*}
    Derivative of the function $f$ with respect to $x_i$
    evaluated in point $(y_1, y_2, \dots, y_n) \in \mathbb{R}^n$ is denoted as follows
    \begin{equation*}
        f^{'}_{x_i} (y_1, y_2, \dots, y_n)
    \end{equation*}
    Moreover, partial derivative $f^{'}_{x_i}$ evaluated at point $(y_1, y_2, \dots, y_n)$ plus
    partial derivative $f^{'}_{x_j}$ evaluated at point $(y_1, y_2, \dots, y_n)$
    is equivalent to the sum of partial derivatives $f^{'}_{x_i}, \; f^{'}_{x_j}$
    evaluated at point $(y_1, y_2, \dots, y_n)$ and to be denoted as
    \begin{equation*}
        f^{'}_{x_i} (y_1, y_2, \dots, y_n) + f^{'}_{x_j} (y_1, y_2, \dots, y_n) = [f^{'}_{x_i} + f^{'}_{x_j}] (y_1, y_2, \dots, y_n)
    \end{equation*}
\end{notation}
Therefore, the following identity in terms of partial derivatives shows the relation between odd-power function $g_{y}$ and
polynomial function $f_{y}$
\begin{thm}
    \label{thm:main-theorem}
    Let be a fixed value $v\in \mathbb{N}$, then derivative $g_v^{'}$ of the odd-power function $g_v(x) = x^{2v + 1}$
    evaluated at point $u$ equals to derivative $(f_{v})^{'}_{x}$ evaluated at point $(u, u)$ plus
    derivative $(f_{v})^{'}_{z}$ evaluated at point $(u, u)$
    \begin{equation}
        g_v^{'} (u) = (f_{v})^{'}_{x} (u, u) + (f_{v})^{'}_{z} (u, u)
        \label{eq:odd-exponential-identity}
    \end{equation}
\end{thm}
Particularly, it follows that for every pair $u, v$ an identity holds
\begin{align*}
(2v+1)
    u^{2v} &= (f_{v})^{'}_{x} (u, u) + (f_{v})^{'}_{z} (u, u) \\
    &= [(f_{v})^{'}_{x} + (f_{v})^{'}_{z}](u,u)
\end{align*}
that is also an ordinary derivative of odd-power function $t^{2v+1}$, therefore
\begin{align*}
    \frac{d}{dt} t^{2v+1} (u) &= (f_{v})^{'}_{x} (u, u) + (f_{v})^{'}_{z} (u, u) \\
    &= [(f_{v})^{'}_{x} + (f_{v})^{'}_{z}](u,u)
\end{align*}
To summarize and clarify all above, we provide a few examples that show an identity~\eqref{eq:odd-exponential-identity}
in action.
\begin{example}
    \normalfont
    Identity~\eqref{eq:odd-exponential-identity} example for $x\in\mathbb{R}, \; z\in \mathbb{R}$ and $y=1$.
    Consider the explicit form of the function $f_{1} (x, z)$ i.e
    \[
        f_1(x, z) = 3 x z - 3 z^2 + 3 x z^2 - 2 z^3
    \]
    Therefore, derivative of $f_{1}$ with respect to $x$ equals to
    \[
        (f_1)^{'}_{x} = \lim_{d \to 0} \frac{3 d z + 3 d z^2}{d} = 3 z + 3 z^2
    \]
    Consider derivative of the function $f_1$ with respect to $z$, that is
    \begin{align*}
    (f_1)
        ^{'}_{z}
        &= \lim_{d \to 0} \left[\frac{-3 d^2 - 2 d^3 + 3 d x + 3 d^2 x - 6 d z - 6 d^2 z + 6 d x z - 6 d z^2}{d} \right] \\
        &= \lim_{d \to 0} \left[ -3 d - 2 d^2 + 3 x + 3 d x - 6 z - 6 d z + 6 x z - 6 z^2 \right] \\
        &=3 x - 6 z + 6 x z - 6 z^2
    \end{align*}
    Combining both $(f_1)^{'}_{x}$ and $(f_1)^{'}_{z}$ evaluated at point $(u, u)$ we get
    \begin{align*}
    (f_1)
        ^{'}_{x} + (f_1)^{'}_{z}
        &= 3 x - 3 z + 6 x z - 3 z^2 \\
        \frac{d}{dt} t^{3} (u) &= [(f_1)^{'}_{x} + (f_1)^{'}_{z}] (u,u)  = 3 u^2
    \end{align*}
    that confirms results of the theorem~\ref{thm:main-theorem}.
\end{example}
\begin{example}
    \normalfont
    Identity~\eqref{eq:odd-exponential-identity} example for $x\in\mathbb{R}, \; z\in \mathbb{R}$ and $y=2$.
    Consider the explicit form of the function $f_{2} (x, z)$ i.e
    \[
        f_2 (x, z) = 5 x^2 z - 15 x z^2 + 15 x^2 z^2 + 10 z^3 - 30 x z^3 + 10 x^2 z^3 + 15 z^4 - 15 x z^4 + 6 z^5
    \]
    Therefore, derivative of $f_{2}$ with respect to $x$ equals to
    \begin{align*}
    (f_2)
        ^{'}_{x} &= \lim_{d \to 0} \left[ 5 d z + 10 x z - 15 z^2 + 15 d z^2 + 30 x z^2 - 30 z^3 + 10 d z^3 +
        20 x z^3 - 15 z^4 \right] \\
        &= 10 x z - 15 z^2 + 30 x z^2 - 30 z^3 + 20 x z^3 - 15 z^4
    \end{align*}
    Consider derivative of the function $f_2$ with respect to $z$, that is
    \begin{align*}
    (f_2)
        ^{'}_{z}
        &= 5 x^2 - 30 x z + 30 x^2 z + 30 z^2 - 90 x z^2 + 30 x^2 z^2 + 60 z^3 - 60 x z^3 + 30 z^4
    \end{align*}
    Combining both $(f_2)^{'}_{x} (x, z)$ and $(f_2)^{'}_{z} (x, z)$ evaluated at point $(u, u)$ we get
    \begin{align*}
    (f_2)
        ^{'}_{x} + (f_2)^{'}_{z} &= 5 x^2 - 20 x z + 30 x^2 z + 15 z^2 - 60 x z^2 + 30 x^2 z^2 + 30 z^3 - 40 x z^3 + 15 z^4\\
        \frac{d}{dt} t^{5} (u) &= [(f_2)^{'}_{x} + (f_2)^{'}_{z}] (u,u) = 5 u^4
    \end{align*}
    that confirms results of the theorem~\ref{thm:main-theorem}.
\end{example}
\begin{example}
    \normalfont
    Identity~\eqref{eq:odd-exponential-identity} example for $x\in\mathbb{R}, \; z\in \mathbb{R}$ and $y=3$.
    Consider the explicit form of the function $f_{3} (x, z)$ i.e
    \begin{align*}
        f_3 (x, z) &= -7 x z + 14 x^2 z + 7 z^2 - 42 x z^2 + 35 x^3 z^2 + 28 z^3 -140 x^2 z^3 + 70 x^3 z^3 + 175 x z^4 \\
        &- 210 x^2 z^4 + 35 x^3 z^4 -70 z^5 + 210 x z^5 - 84 x^2 z^5 - 70 z^6 + 70 x z^6 - 20 z^7
    \end{align*}
    Therefore, derivative of $f_{3}$ with respect to $x$ equals to
    \begin{align*}
    (f_3)
        ^{'}_{x} &= -7 z + 28 x z - 42 z^2 + 105 x^2 z^2 - 280 x z^3 + 210 x^2 z^3 + 175 z^4 - 420 x z^4 \\
        &+ 105 x^2 z^4 + 210 z^5 - 168 x z^5 + 70 z^6
    \end{align*}
    Consider derivative of the function $f_2$ with respect to $z$, that is
    \begin{align*}
    (f_3)
        ^{'}_{z} &= -7 x + 14 x^2 + 14 z - 84 x z + 70 x^3 z + 84 z^2 - 420 x^2 z^2 + 210 x^3 z^2 + 700 x z^3 \\
        &- 840 x^2 z^3 + 140 x^3 z^3 - 350 z^4 + 1050 x z^4 - 420 x^2 z^4 - 420 z^5 + 420 x z^5 - 140 z^6
    \end{align*}
    Combining both $(f_3)^{'}_{x} (x, z)$ and $(f_3)^{'}_{z} (x, z)$ evaluated at point $(u, u)$ we get
    \begin{align*}
    (f_3)
        ^{'}_{x} + (f_3)^{'}_{z} &= -7 x + 14 x^2 + 7 z - 56 x z + 70 x^3 z + 42 z^2 - 315 x^2 z^2 + 210 x^3 z^2 \\
        &+ 420 x z^3 - 630 x^2 z^3 + 140 x^3 z^3 - 175 z^4 + 630 x z^4 - 315 x^2 z^4 - 210 z^5 \\
        &+ 252 x z^5 - 70 z^6 \\
        \frac{d}{dt} t^{3} (u) &= [(f_3) ^{'}_{x} + (f_3)^{'}_{z}] (u,u) = 7 u^6
    \end{align*}
    that confirms results of the theorem~\ref{thm:main-theorem}.
\end{example}